\documentclass[a4paper,12pt]{article}

\usepackage[utf8]{inputenc}
\usepackage[brazil]{babel}
\usepackage{indentfirst}

\title{Aula latex}
\author{Fernando Mayer}
\date{\today}

\begin{document}

\maketitle
\tableofcontents

\newpage

\section{Introdução}

A primeira aula sobre latex.

Vamos ver alguns comandos basicos abaixo.

O LaTeX é um {\bf conjunto} de macros para o processador de textos TeX, utilizado amplamente para a produção de textos
{\huge matemáticos} e \textbf{científicos} devido à sua alta qualidade
tipográfica. Entretanto, também é utilizado para        produção
de cartas pessoais, artigos e livros sobre assuntos
diversos.

\begin{verbatim}
O LaTeX é um {\bf conjunto} de macros
para o processador de textos TeX, 
utilizado amplamente para a produção de textos
{\huge matemáticos} e 
\textbf{científicos} devido à sua alta qualidade

tipográfica. Entretanto, também é utilizado para        produção
de cartas pessoais, artigos e livros sobre assuntos
diversos.
\end{verbatim}

% Alguma coisa 10%
Alguma coisa 10\%

\section{Metodologia}

\subsection{Dados}

Os dados coletados foram:

\begin{itemize}
\item expVida: expectativa de vida
\item pibPercap: PIB per capta
\end{itemize}

Os dados foram importados com  comando:

\begin{verbatim}
dados <- read.table("pib_gapminder.csv",
                    header = TRUE, sep = ",", dec = ".")
\end{verbatim}

Este conjunto de dados tem 1114 linhas e 5 colunas.

\subsection{Análise}

As análises foram:

\begin{enumerate}
\item Análise exploratória
\item Associação entre variáveis
\begin{itemize}
\item pibPercap e expVida
\end{itemize}
\end{enumerate}

\end{document}





























